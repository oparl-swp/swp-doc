\documentclass{beamer}

\usepackage[ngerman]{babel}
\usepackage[T1]{fontenc}
\usepackage{graphicx}
\usepackage{verbatim}
\usepackage{mdwlist}
\usepackage{listings}
\usepackage[notes, backend=biber]{biblatex-chicago}
\bibliography{gismondine}

\usepackage{xunicode}
\usepackage{xltxtra}
\defaultfontfeatures{Mapping=tex-text}
\setmonofont[Mapping={}, Scale=MatchLowercase]{DejaVu Sans Mono}
\setsansfont[Scale=MatchLowercase]{Linux Biolinum O}
\setmainfont[]{Linux Libertine O}

\newbox\mytempbox
\newdimen\mytempdimen
\newcommand\includegraphicscopyright[3][]{%
  \leavevmode\vbox{\vskip3pt\raggedright\setbox\mytempbox=\hbox{%
  \includegraphics[#1]{#2}}%
    \mytempdimen=\wd\mytempbox\box\mytempbox\par\vskip1pt%
    \fontsize{3}{3.5}\selectfont{\color{black!25}{\vbox{\hsize=\mytempdimen#3}}}\vskip3pt%
}}

\newcommand\prelim[1]{\small }

\newcommand\strColor[1]{\color{beamer@blendedblue}{#1}}

\newcommand\sect[1]{\begin{center}\huge\strColor{#1}\end{center}}

\setbeamerfont{page number in head/foot}{size=\large}
\setbeamertemplate{navigation symbols}{}
\setbeamertemplate{headline}
{%
    \begin{beamercolorbox}{section in head/foot}
        \vskip1em
        \insertsection\hfill\includegraphics[height=5em]{FULogo_RGB.eps}\hspace{1em}
        \vskip-5.3em
    \end{beamercolorbox}%
}
\setbeamertemplate{footline}[frame number]

\title{OParl-Validator}
\author{Das OParl-Validator-Team}
\institute{Freie Universität Berlin\\Institut für Informatik}
\date{2. Juli 2014}

\begin{document}

\frame{\titlepage}

%\frame{\tableofcontents}

\frame{
    \frametitle{Projekt-Struktur}
    \begin{itemize}
        \item Unsere Kunden
        \item Unsere Arbeitsweise
        \item Mockup und erste Diskussionen
        \item Methodologische Anforderungen
        \item Server Anforderungen
        \item Unsere konstruktiven Beiträge zur Spezifikation
        \item Demonstration
    \end{itemize}
}

\frame{
    \frametitle{Methodologische Anforderungen}
    \begin{itemize}
        \item JSON Schema
        \item PEP 8
        \item Coverage
        \item Travis CI
        \item OParl PyPI account
    \end{itemize}
}

\frame{
    \frametitle{Server Requirements}
    \begin{itemize}
        \item Liste der Anforderungen
        \item JSON Schema
        \item proprietäre Erweiterungen
            \lstinputlisting[basicstyle=\footnotesize\tt]{schema-snippet.json}
    \end{itemize}
}


\frame{
    \frametitle{Unsere konstruktiven Beiträge zur Spezifikation}
    \begin{itemize}
        \item Häufige Änderungen an der Spezifikation trotz Reviewphase
        \item Kleine Änderungen: 4 Pull Requests, 4 Issues
        \item Feedback zur Verwendung von JSON-LD gegeben
}

\frame{
    \frametitle{JSON-LD}
    \begin{verbatim}
{
    "@context": [
        {
            "oparl": "http://oparl.org/specs/1.0/schema/",
            "rdfs": "http://www.w3.org/TR/rdf-schema/#",
            "body": {
              "@id": "oparl:body",
              "@type": "@id"
            },
            "name": "rdfs:label"
        }
    ],
    "@type": "oparl:Meeting",
    "body": "http://ris.exmaple.org/body/0",
    "name": "Erste Sitzung"
}
    \end{verbatim}
    \begin{verbatim}
    [
  {
    "@type": [
      "http://oparl.org/specs/1.0/schema/Meeting"
    ],
    "http://oparl.org/specs/1.0/schema/body": [
      {
        "@id": "http://ris.exmaple.org/body/0"
      }
    ],
    "http://www.w3.org/TR/rdf-schema/#label": [
      {
        "@value": "Erste Sitzung"
      }
    ]
  }
]

}   
    \end{verbatim}
\end{document}
