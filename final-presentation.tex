\documentclass{beamer}

\usepackage[ngerman]{babel}
\usepackage[T1]{fontenc}
\usepackage{graphicx}
\usepackage{verbatim}
\usepackage{mdwlist}
\usepackage{listings}
\usepackage{ragged2e}

\usepackage{xunicode}
\usepackage{xltxtra}
\defaultfontfeatures{Mapping=tex-text}
\setmonofont[Mapping={}, Scale=MatchLowercase]{DejaVu Sans Mono}
\setsansfont[Scale=MatchLowercase]{Linux Biolinum O}
\setmainfont[]{Linux Libertine O}

\newbox\mytempbox
\newdimen\mytempdimen
\newcommand\includegraphicscopyright[3][]{%
  \leavevmode\vbox{\vskip3pt\raggedright\setbox\mytempbox=\hbox{%
  \includegraphics[#1]{#2}}%
    \mytempdimen=\wd\mytempbox\box\mytempbox\par\vskip1pt%
    \fontsize{3}{3.5}\selectfont{\color{black!25}{\vbox{\hsize=\mytempdimen#3}}}\vskip3pt%
}}

\let\raggedright=\RaggedRight
\hyphenation{Freeze}

\newcommand\prelim[1]{\small }

\newcommand\strColor[1]{\color{beamer@blendedblue}{#1}}

\newcommand\sect[1]{\begin{center}\huge\strColor{#1}\end{center}}

\setbeamerfont{page number in head/foot}{size=\large}
\setbeamertemplate{navigation symbols}{}
\setbeamertemplate{headline}
{%
    \begin{beamercolorbox}{section in head/foot}
        \vskip1em
        \insertsection\hfill\includegraphics[height=5em]{FULogo_RGB.eps}\hspace{1em}
        \vskip-5.3em
    \end{beamercolorbox}%
}
\setbeamertemplate{footline}[frame number]

\title{OParl-Validator}
\author{Das OParl-Validator-Team}
\institute{Freie Universität Berlin\\Institut für Informatik}
\date{$n$. Oktober 2014}

\begin{document}

\frame{\titlepage}

%\frame{\tableofcontents}

\frame{
    \frametitle{Gliederung}
    \begin{itemize}
        \item Organisation
        \item Architektur
        \item Features
        \item Close-Out-Plan
        \item Demo
    \end{itemize}
}

\frame{
    \frametitle{Organisation}
    \begin{itemize}[<+->]
        \item Prozesse
        \begin{itemize}
            \item Kanban auf Trello
            \item Continuous Integration
        \end{itemize}
        \item Meetings
        \begin{itemize}
            \item Mumble und Hangout
            \item Häufig mit Stakeholdern
        \end{itemize}
        \item Hackathons bei Spline
        \item Protokolle
    \end{itemize}
}

\frame{
    \frametitle{Architektur}
    \begin{columns}
        \begin{column}{.5\textwidth}
            \begin{itemize}[<+->]
                \item Validator
                \begin{itemize}[<*>]
                    \item Basierend auf Freeze vom 22. Juli
                    \item Mit einigen optionalen Features
                \end{itemize}
                \item Crawler
                \begin{itemize}[<*>]
                    \item Validiert ein komplettes System
                    \item Oder einen Teil des Systems
                \end{itemize}
                \item Server Tests
                \begin{itemize}[<*>]
                    \item Suite mit Anforderungen auf tieferer Abstraktionsebene
                    \item Aber auf Layer 7
                \end{itemize}
                \item Command Line Interface
                \begin{itemize}[<*>]
                    \item Gemeinsamer Entry-Point
                \end{itemize}
            \end{itemize}
        \end{column}
        \begin{column}{.5\textwidth}
            \vspace{-.5cm}
            \includegraphics[width=\textwidth]{architecture.final.png}
        \end{column}
    \end{columns}
}

\frame{
    \frametitle{Features}
    \begin{itemize}[<+->]
        \item Continuous Integration
        \begin{itemize}[<*>]
            \item Dank Travis CI
            \item Mit automatischer Coverage dank Coveralls
        \end{itemize}
        \item Python 2.7 und 3.3
        \begin{itemize}[<*>]
            \item \texttt{six}-Library sehr hilfreich
            \item Continuous Integration ebenso
        \end{itemize}
        \item Back-off Modi
        \begin{itemize}[<*>]
            \item Was meinte ich denn hiermit?
        \end{itemize}
        \item Constraints
        \begin{itemize}[<*>]
            \item Type Whitelist, Rekursion, etc.
        \end{itemize}
    \end{itemize}
}

\frame{
    \frametitle{Close-Out-Plan}
    \begin{itemize}[<+->]
        \item OParl 1.0 noch nicht finalisiert
        \item Einige des Teams werden die letzten Anpassungen zu gegebener Zeit vornehmen
        \item Einige des Teams werden das Projekt auch langfristig zusammen warten
    \end{itemize}
}
\end{document}
