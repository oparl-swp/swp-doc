\documentclass[12pt, fleqn]{scrartcl}

\usepackage[singlecolumn, german]{fnordstyle}
\usepackage{hyperref}
\usepackage{graphicx}
\usepackage{picins}
\usepackage{wrapfig}
\usepackage{verbatim}
\usepackage{tabularx}
\usepackage{csquotes}
\usepackage[style=mla, backend=biber]{biblatex}
\bibliography{oparl}
%\typearea{11}

\setlength{\abovecaptionskip}{-1em}
\setlength{\belowcaptionskip}{1em}
\let\oldnewcaption\newcaption
\renewcommand{\newcaption}[1]{
    \vspace{-1em}
    \oldnewcaption{#1}
    \vspace{1em}
}

\setcounter{tocdepth}{2}

\newcommand{\autocitenp}[2][]{\citeauthor{#2} \autocite*[#1]{#2}}

\hyphenation{mne-mon-ic}

\title{OParl-Validator-Protokoll}
%\subtitle{Flauschnotizen}
\author{Alex, Jo, Lucas, Ronald, Telofy}
\date{\today}

\begin{document}

\Maketitle

\section{Einführung}
\section{Organisation}
\section{Features}
\section{Demo}
\section{Architektur}

\begin{wrapfigure}{R}{0.5\textwidth}
  \begin{center}
    \includegraphics[width=.5\textwidth]{architecture.final.png}
  \end{center}
  \caption{Architektur}
\end{wrapfigure}

\subsection{Idee}

Die Grundidee hinter der Architektur unseres Projektes basiert auf dem Mantra, dass, wenn man ein Command-Line-Tool entwickeln möchte, man erst eine Library entwickeln soll, welche die Arbeit tut, und dann ein Command-Line-Tool als Frontend für sie schreiben soll.

Das ist eine sehr nützliche Vorgehensweise, da sie alle Vorteile von Command-Line-Tools beibehält, während sie es Entwicklern ermöglicht, von der Flexibilität der eigentlichen Library zu profitieren. Git, zum Beispiel, basiert auf dieser Architektur.

In unserem Fall ist es so, dass es von Anfang an unser Anspruch war, beides, eine Library und ein Command-Line-Tool, zu entwickeln. Letzteres ist im selben Projekt aufgehoben, die Library aber kann auch mit anderen Frontends verwendet werden.

Die Validator-Library sollte zudem sowohl die Fähigkeit mitbringen, JSON-Daten direkt von der Command Line zu validieren, als auch die Standardfunktionalität, URLs eines OParl-Systems zu validieren. Zu diesem Zweck hat der Command-Line-Client zwei Code-Pfade, von denen einer den Validator direkt aufruft, während der zweite eine URL an einen Crawler als Seed übergibt.

\subsection{JSON-Validator}

Der Validator durchschreitet in seiner Tätigkeit verschiedene Phasen, zum Teil notwendiger Weise, und zum Teil, um bessere Fehlermeldungen ausgeben zu können. Hierbei mussten wir stets die Abwägung treffen, zu welchem Maße wir Leanness dem Komfort zu opfern bereit sind, den wir erwarten, die Benutzer° wertschätzen werden.

Zunächst wird das JSON selbst auf Fehler überprüft. Solche würden eine Weiterverarbeitung verhindern. Dann testet der Validator, ob ein gültiger Typ im Objekt spezifiziert ist. Ist das nicht der Fall, so können wir nicht schließen, welchem Format das Objekt folgen soll, und eine weitere Validierung ist auch ausgeschlossen.

Nach erfolgreichen Passierens dieser Phasen des Tests, kommt das JSON-Schema zum Einsatz, in das wir die Spezifikation übersetzt haben. Dieses Schema bildet das Herzstück des Validators, und ist versioniert, um mögliche zukünftige Versionen des Standards parallel unterstützen zu können. Alle Bestandteile des Schemas, die sich als JSON-Schema ausdrücken lassen, sind in diesen Dateien kodifiziert.

Letztlich gibt es noch einige Teile des Standards, die sich nicht im JSON-Schema abbilden lassen. Diese sind als Python-Funktionen implementiert und, mittels proprietärer Erweiterungen, im Schema referenziert.

\subsection{Response-Validator}

Neben dieser Kette an Validierungsschritten, gib es außerdem ein Validierungssystem für HTTP-Responses. Der Standard macht auch Vorschriften über die Headers, die ein Server ausliefern soll. Wenn der Validator also verwendet wird, um Antworten eines OParl-Servers zu validieren, so stellt dieser Teil des Systems sicher, dass sie Standardkonform sind.

\subsection{Server-Tests}

Es gibt eine Reihe weiterer Features von OParl, die es erfordern, dezidierte Requests zu senden, die zur Validierung der JSON-Objekte redundant wären. Unsere Test-Suite bedient sich an einer Liste von validen URLs von Crawler (der im Folgenden näher erläutert werden wird), und benutzt diese, um generische Eigenschaften und Fähigkeiten des Servers einer Prüfung zu unterziehen.

\subsection{Crawler}

% Outtake: Wenn das JSON-Schema das Herzstück unseres Validators ist, dann ist der Crawler Haut und Brustbehaarung.
% Outtake: Wenn das JSON-Schema das Herzstück unseres Validators ist, dann isst der Crawler seine Extremitäten.

Wenn das JSON-Schema das Herzstück unseres Validators ist, dann ist der Crawler seine Extremitäten und sein Schweif.



\section{Close-Out-Plan}

%\newpage

\phantomsection
\addcontentsline{toc}{section}{References}
\setlength\bibitemsep{0pt}
\printbibliography

\end{document}

Count
    
    grep -vE '^[\\ %}]' protocol.tex | wc
